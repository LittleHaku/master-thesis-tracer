% !TeX root = ../main.tex
% Add the above to each chapter to make compiling the PDF easier in some editors.

\chapter{Background and State of the Art}\label{chapter:state_of_the_art}

%TODO: Hacer una intruducción al capítulo

\section{Background}

\subsection{Conversational Agents}

Conversational agents, commonly referred to as chatbots,
are software systems designed to interact with users through natural language dialogue.
These systems have evolved from simple rule-based programs that followed predefined conversation flows
to sophisticated AI-powered agents capable of understanding context, maintaining conversational state, and generating human-like responses.

Modern conversational agents can be categorized into two main types
given the domain and range of their capabilities.
\begin{itemize}
  \item \textbf{Task-oriented:} on one hand we have the task-oriented chatbots,
  these are designed to assist users in completing specific tasks, such as booking appointments, processing orders, or providing customer support.
  These systems typically follow structured conversation flows
and maintain explicit state management to track task progress.
  Examples of these chatbots are Taskyto \autocite{sanchezcuadradoAutomatingDevelopmentTaskoriented2024} or UAM's assistant Ada \autocite{AdaUAM}.

  \item \textbf{Open-domain:}
  on the other hand, open-domain chatbots
  engage in general conversation without specific task constraints,
  aiming to provide informative, helpful, or entertaining interactions across a wide range of topics.
  These are chatbots like ChatGPT \autocite{ChatGPT} or Gemini \autocite{GoogleGemini}.
\end{itemize}

The development of these conversational agents
has beeen facilitated by various frameworks and platforms.
\begin{itemize}
  \item \textbf{Intent-based frameworks:}
    these frameworks such as Google's Dialogflow \autocite{Dialogflow} or Rasa \autocite{Rasa2020}
    enable developers to define conversation flow through intents, utterances, and responses.
    These platforms have low latency and deterministic behavior
    but are very rigid, struggle to scale,
    and to work properly require a big corpus to be trained on.

  \item \textbf{Multi-agent programming environments:}
    these systems like LangGraph \autocite{LangGraph} or Microsoft's AutoGen \autocite{AutoGen},
    allow for the creating of complex conversational systems
    where multiple \ac{AI} agents collaborate to process the user's request.
    These frameworks make use of the capabilities of \acp{LLM}.
    While they are less rigid than the previous ones,
    they can suffer from hallucinations, 
    higher latency, and since they are not deterministic,
    getting out of the scope, and thus, making it harder to test it.
\end{itemize}


