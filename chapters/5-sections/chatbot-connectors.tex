\section{Chatbot Connectors}\label{sec:chatbot-connectors}

A key component of \ac{TRACER} is its connector system,
developed as part of this work and
packaged as the \texttt{chatbot-connectors} library.
The \texttt{chatbot-connectors} library is available on PyPI
(\url{https://pypi.org/project/chatbot-connectors/})
and its source code can be accessed at
\url{https://github.com/Chatbot-TRACER/chatbot-connectors}.

The purpose of this package is to provide a unified interface
to interact with different chatbots,
having the same method for sending messages
but different implementations for each chatbot technology.
For this we have an abstract base class
from which every implementation inherits.
This base class standardizes the core methods
for sending messages and receiving responses.

\subsection{Available Connector Technologies}

\ac{TRACER} currently supports four connector technologies,
each designed to interface with specific chatbot platforms:
Custom (a flexible YAML-configured connector for any chatbot \ac{API}),
MillionBot (specialized for the MillionBot platform),
Rasa (interfacing with Rasa chatbots through \ac{REST} webhooks),
and Taskyto (dedicated connector for Taskyto chatbot servers).

The Custom connector deserves special attention due to its flexibility,
providing a solution for integrating with any chatbot \ac{API}
without requiring custom code development.
This connector uses YAML configuration files
to define how to communicate with a chatbot's \ac{API},
specifying the \ac{API} endpoint, request structure,
authentication requirements, and response parsing instructions.

\subsection{Connector Discovery and Configuration}

The \ac{CLI} provides built-in commands to discover available connectors
and their configuration requirements:

\begin{lstlisting}[
language=tracer-examples,
caption={Listing available connectors.},
label={code:list-connectors}
]
$ tracer --list-connectors
Available Chatbot Connector Technologies:
==================================================
  
  - custom
    Description: Custom chatbot connector configured by a YAML file
    Use: --technology custom
    Parameters: --list-connector-params custom

  - millionbot
    Description: MillionBot chatbot connector
    Use: --technology millionbot
    Parameters: --list-connector-params millionbot

  - rasa
    Description: RASA chatbot connector using REST webhook
    Use: --technology rasa
    Parameters: --list-connector-params rasa

  - taskyto
    Description: Taskyto chatbot connector
    Use: --technology taskyto
    Parameters: --list-connector-params taskyto

Total: 4 connector(s) available
\end{lstlisting}

For detailed parameter information for a specific connector,
users can query the requirements:

\begin{lstlisting}[
language=tracer-examples,
caption={Listing connector parameters for Taskyto.},
label={code:list-connector-params}
]
$ tracer --list-connector-params taskyto
Parameters for 'taskyto' chatbot:
--------------------------------------------------
  - Name: base_url
    Type: string
    Required: True
    Description: The base URL of the Taskyto server.

  - Name: port
    Type: integer
    Required: False
    Default: 5000
    Description: The port of the Taskyto server.

Example usage:
  JSON format: --connector-params '{"base_url": "http://localhost"}'
  Key=Value format: --connector-params "base_url=http://localhost"
\end{lstlisting}

\subsection{Custom YAML Connector Configuration}

The Custom connector deserves special attention due to its flexibility.
It allows users to integrate with any chatbot \ac{API}
by creating a YAML configuration file that describes
the \ac{API} communication protocol.

A Custom connector configuration file contains the following key fields:

\begin{itemize}
    \item \texttt{name}: A friendly name for the chatbot (optional)
    \item \texttt{base\_url}: The base \acs{URL} of the chatbot \ac{API} (required)
    \item \texttt{send\_message.path}: \ac{API} endpoint path appended to base\_url (required)
    \item \texttt{send\_message.method}: \acs{HTTP} method (POST, GET, PUT, DELETE; default: POST)
    \item \texttt{send\_message.headers}: Custom headers including authentication (optional)
    \item \texttt{send\_message.payload\_template}: \acs{JSON} structure with \texttt{\{user\_msg\}} placeholder (required)
    \item \texttt{response\_path}: Dot-separated path to extract the bot's reply from the \acs{JSON} response (required)
\end{itemize}

\subsubsection{Configuration Examples}

A simple echo bot configuration demonstrates the basic structure:

\begin{lstlisting}[
language=connector-examples,
caption={Simple Custom connector configuration.},
label={code:simple-yaml-config}
]
name: "Echo Bot"
base_url: "https://postman-echo.com"
send_message:
  path: "/post"
  method: "POST"
  payload_template:
    message: "{user_msg}"
response_path: "json.message"
\end{lstlisting}

A more complex configuration with authentication:

\begin{lstlisting}[
language=connector-examples,
caption={Custom connector configuration with authentication.},
label={code:auth-yaml-config}
]
name: "Secure Bot"
base_url: "https://api.mychatbot.com"
send_message:
  path: "/chat/send"
  method: "POST"
  headers:
    Authorization: "Bearer your-api-key"
    Content-Type: "application/json"
  payload_template:
    query: "{user_msg}"
    session_id: "user123"
response_path: "response.text"
\end{lstlisting}

The \texttt{response\_path} field uses dot notation to navigate through \ac{JSON} responses:
\begin{itemize}
  \item \texttt{"message"} accesses \texttt{response["message"]}
  \item \texttt{"data.text"} accesses \texttt{response["data"]["text"]}
  \item \texttt{"results.0.content"} accesses
\texttt{response["results"][0]["content"]}
\end{itemize}
