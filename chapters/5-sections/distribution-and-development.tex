\section{Distribution and Development Workflow}

Before detailing the \ac{CLI}'s functioning,
this section will describe \ac{TRACER}'s packaging, distribution,
and the software engineering practices used to maintain its quality.
\ac{TRACER} is packaged and distributed as a package on the \acf{PyPI} repository
(\url{https://pypi.org/project/chatbot-tracer/}),
facilitating its installation by running
\texttt{pip install chatbot-tracer}.
This approach not only simplifies its use,
but also its implementation into other projects
such as the web application that was developed,
or other projects that could use \ac{TRACER}
since it can be added as another package requirement.

To ensure code quality and automate the release process
\ac{TRACER} makes use of GitHub Actions for the \ac{CI/CD} pipeline.
For the \ac{CI} we used Ruff \autocite{Ruff}.
Ruff is Python linter and formatter written in Rust
that combines tools like Flake8 or Black into a single and faster tool.
We used Ruff not only to enforce a consistent code style,
but also to enforce code quality standards,
such as ensuring proper documentation
and managing code complexity by setting thresholds for metrics like McCabe's cyclomatic complexity.
For the \ac{CD} side,
we implemented a pipeline that whenever a tag with the format \texttt{v*.*.*} is published
automatically builds the package,
publishes it to \ac{PyPI},
and creates the corresponding GitHub release.
All the \ac{TRACER} source code can be accessed in \url{https://github.com/Chatbot-TRACER/TRACER}.
