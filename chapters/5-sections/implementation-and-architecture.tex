\section{Implementation and Architecture}

\subsection{Core Framework: LangGraph}

As explained during the \autoref{sec:profile-generation} (\nameref{sec:profile-generation})
\ac{TRACER} relies on \acp{LLM},
this is why we used LangGraph \autocite{LangGraph} as our framework for development.
We chose LangGraph because
it allows to manage and orchestrate
complex agentic workflows with states,
it is also an industry standard with extensive documentation.
LangGraph allows us to orchestrate the different stages
and to keep complex states where we store
the inferred model and the fields generated for the profiles.
Currently, \ac{TRACER} allows OpenAI and Gemini models,
but thanks to LangGraph it would be straightforward to add other \ac{LLM} providers.

\subsection{Modular Architecture}

\ac{TRACER} can infer a chatbot model as long as
the chatbot is accessible through an interface, typically a \acs{REST} \acs{API}.
Currently, it provides access to communicate with chatbots made with different technologies,
such as Taskyto \autocite{sanchezcuadradoAutomatingDevelopmentTaskoriented2024}, Rasa \autocite{Rasa2020} or 1MillionBot \autocite{1MillionBot}.
In addition, new connectors could be added by extending the current implementation.

Apart from these connectors,
\ac{TRACER} is divided into three modules,
each corresponding to a phase of the methodology:

\begin{itemize}
  \item \textbf{Explorer Module:}
    Contains the Explorer Agent
    and implements the logic for the Exploration Phase (see \autoref{sec:exploration}),
    managing the conversational sessions and the initial extraction of Functionality Nodes.

  \item \textbf{Refinement Module:}
    Implements the logic for the Refinement Phase (see \autoref{sec:refinement}),
    responsible for consolidating functionalities,
    classifying the chatbot, and inferring the final workflow structure.

  \item \textbf{Profile Generation Module:}
    Implements the seven-step synthesis process (see \autoref{sec:profile-generation}),
    taking the final chatbot model and generating the YAML user profiles.
\end{itemize}

\subsection{Generated Artefacts}

Upon completion, \ac{TRACER} generates the following artefacts
containing the results from the full analysis performed on the target chatbot:

\begin{itemize}
  \item A set of user profiles
    representing realistic users that would use the application
    and that will act as test cases
    (see \autoref{code:yaml-profile-pizza} and \autoref{code:yaml-profile-drinks})
  \item A markdown report containing the inferred model information such as
    the discovered functionalities, fallback message, language
    and other information such as token usage,
    number of \ac{LLM} calls or estimated cost.
  \item A graph representing the inferred model's workflow (see \autoref{fig:pizzeria-workflow})
  \item A JSON file containing the same workflow but in a textual format.
\end{itemize}
