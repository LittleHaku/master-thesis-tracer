% !TeX root = ../main.tex
% Add the above to each chapter to make compiling the PDF easier in some editors.

\chapter{Conclusions and Future Work}\label{chapter:conclusion}

This final chapter synthesises the contributions of the thesis,
summarises the key findings from the experimental evaluation,
and outlines promising directions for future research.

\section{Conclusions}

The proliferation of complex, heterogeneous, and often black-box conversational agents
has created a pressing need for advanced, automated testing methodologies.
As established in this thesis, existing approaches to chatbot quality assurance
are frequently limited by their reliance on manual effort,
access to source code, or pre-existing conversation corpora,
hindering their applicability in real-world scenarios.

This thesis presented \ac{TRACER}, a two-phase methodology
for the automated exploration and profiling of conversational agents.
The approach begins with an Exploration Phase,
where an \ac{LLM}-powered agent systematically interacts with a chatbot
to automatically infer a functional model of its capabilities.
This is followed by a Profile Generation process
that synthesizes a comprehensive suite of detailed, executable test user profiles
directly from the inferred model.
The entire methodology is implemented as an open-source tool,
available via a \ac{CLI} or through a web application.

The effectiveness of this approach
was rigorously validated through three research questions.
The evaluation first demonstrated that in a controlled, white-box environment,
\ac{TRACER} is highly effective
at achieving high functional coverage of a chatbot's capabilities (RQ1).
Furthermore, the synthesised profiles proved capable of detecting faults,
successfully identifying an average of 84.6\% of injected errors
in a comprehensive mutation testing analysis (RQ2).
Finally, in a practical, black-box setting against five real-world deployed chatbots,
the models inferred by \ac{TRACER} were shown to be highly accurate,
achieving an average precision of 96.0\% upon manual verification (RQ3).

Taken together, these findings confirm
that the methodology presented in this thesis
provides a practical and effective solution
that significantly advances the state of the art.
By successfully bridging the gap
between automated model inference and test case synthesis,
\ac{TRACER} offers a fully automated, framework-agnostic, and black-box approach
to the quality assurance of modern conversational agents.

\section{Future Work}

While this thesis presents a complete and validated methodology,
this research opens up several promising avenues for future work.
The following directions could build upon the foundation established by \ac{TRACER}:

\begin{itemize}
    \item \textbf{Improving the Consolidation Logic:}
      The evaluation of RQ3 revealed that while TRACER's discovery process is robust,
      the automated consolidation of semantically similar functionalities remains a challenge,
      especially for highly complex chatbots.
      Future work could explore more sophisticated semantic clustering algorithms. 

    \item \textbf{Enhancing Profile Synthesis with Advanced Testing Strategies:}
      The current profile generation process creates comprehensive tests for discovered functionalities.
      A promising extension would be to incorporate more advanced and adversarial testing strategies.
      This could include synthesizing profiles that specifically test for security vulnerabilities like prompt injection,
      check for social biases in the chatbot's responses,
      or simulate users who are deliberately uncooperative or attempt to derail the conversation.

    \item \textbf{Extending to Multimodal and Voice-Based Agents:}
      The current implementation of \ac{TRACER} focuses on text-based conversational agents.
      A significant future direction would be
      to extend the methodology to support multimodal chatbots that interact using images, interactive buttons, and voice.
      This would require extending the `chatbot-connectors` library
      to handle different input/output formats
      and leveraging multimodal \acp{LLM} for both the Explorer Agent and the user simulator.

    \item \textbf{Quantitative Measurement of Recall in Black-Box Settings:}
      As identified in the threats to validity for RQ3,
      a major open challenge in the field is the quantitative measurement of recall (completeness) for black-box systems.
      Future research could investigate the application of statistical methods,
      such as capture-recapture models from ecology,
      to estimate the total number of functionalities in a system.
      This could provide a quantitative estimate of recall,
      even without access to the source code,
      further strengthening the evaluation of black-box discovery tools.
\end{itemize}

These future directions highlight the potential for the \ac{TRACER} methodology
to serve as a foundational platform
for a new generation of intelligent, automated quality assurance tools
for the rapidly evolving landscape of conversational AI.


