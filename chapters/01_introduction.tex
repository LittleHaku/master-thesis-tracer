% !TeX root = ../main.tex
% Add the above to each chapter to make compiling the PDF easier in some editors.

\chapter{Introduction}\label{chapter:introduction}



The growth of conversational agents, popularly known as chatbots,
has changed the way humans interact with computers across a range of domains.
From general-purpose assistants like OpenAI's ChatGPT \autocite{ChatGPT} or Google's Gemini \autocite{GoogleGemini} to task-oriented agents that help users in particular tasks such as shopping or customer service.
Such systems provide natural language interaction with services from customer service and e-commerce websites to educational materials.
The spread of these agents has also been boosted by developments in generative \ac{AI}, particularly \acp{LLM}, which have dramatically improved chatbot functionality, enabling them to both generate and comprehend natural language without explicitly programmed rules.

The fact that they appear in so many uses has increased the concern about their correctness, reliability, and quality assurance.
As these systems become ubiquitous in areas like healthcare or finance, which demand levels of trust that are high, the requirement for validation and testing becomes paramount.
Nevertheless, the heterogeneousness of chatbot building, with intent-based platforms such as Google's Dialogflow \autocite{Dialogflow} or Rasa \autocite{Rasa2020}, multi-agent programming environments based on \acp{LLM} like LangGraph \autocite{LangGraph} and Microsoft's AutoGen \autocite{AutoGen}, and \acp{DSL} such as Taskyto \autocite{sanchezcuadradoAutomatingDevelopmentTaskoriented2024}, imposes great difficulties in seeking an overarching methodology to test these systems.


Conventional software testing methods are hardly applicable to chatbot systems.
The intricacy of \ac{NLP}, the non-deterministic nature of \acp{LLM} and the dynamic flow of a real conversation make traditional testing insufficient for dialogue agents.
Although there have been some methods for developing testing methods for chatbots \cite{cuadradoIntegratingStaticQuality2024, canizaresMeasuringClusteringHeterogeneous2024}, they often focus on particular chatbot technologies \autocite{RasaTest2025}, require substantial manual effort including the provision of test conversations \autocite{CyaraBotium, RasaTest2025} or synchronous human interaction \autocite{renEvaluationTechniquesChatbot2019}, rely on available conversation corpus \autocite{vasconcelosBottesterTestingConversational2017}, or require access to the source code of the chatbot \autocite{canizaresCoveragebasedStrategiesAutomated2024, gomez-abajoMutationTestingTaskOriented2024, urricoMutaBotMutationTesting2024}, thus restricting their applicability to deployed systems as black boxes.


The work in this thesis seeks to address these issues by the development of Task Recognition And Chatbot ExploreR (\ac{TRACER}), a tool for extracting comprehensive models from deployed conversational agents, and then, with this model, generate user profiles that are test cases for a user simulator named Sensei \autocite{delaraAutomatedEndtoEndTesting2025, delaraSensei}.
\ac{TRACER} uses an \ac{LLM} agent to systematically investigate the chatbot's abilities through natural language interactions, without requiring manual test case writing or access to the source code of the chatbot.
This black-box strategy facilitates automated generation of comprehensive chatbot models that capture supported languages, fallback mechanisms, functional capabilities, input parameters, acceptable parameter values, output data structures, and conversational flow patterns.


The extracted chatbot model serves as the foundation for the automated synthesis of test cases.
In particular, \ac{TRACER} produces user profiles that model varied users that interacts with the chatbot through Sensei \autocite{delaraAutomatedEndtoEndTesting2025, delaraSensei}, yet alternate implementations of \ac{TRACER} could be used to produce various kinds of test cases from the extracted model.
The combination of \ac{TRACER} and Sensei results in a test approach that requires just a connector for the chatbot’s API.

In order to make this research accessible and reproducible, \ac{TRACER} has been developed as a full, open-source tool.
It is available publicly as a \ac{PyPI} package \autocite{sotillodelhornoChatbottracerToolModel} and can be installed using \texttt{pip install chatbot-tracer}.
The complete source code is available on GitHub \url{https://github.com/Chatbot-TRACER/TRACER}, and a special web application has been created to offer an easy experience for the whole test pipeline, ranging from model extraction and user profiles generation with \ac{TRACER} to test execution with Sensei.


To direct this inquiry, we have established the following research questions:
\begin{itemize}
\item \textbf{RQ1: How effective is TRACER in modeling chatbot functionality?}
  This question evaluates the capability of our model discovery method to attain high functional coverage in a controlled environment where the ground truth is available.
\item \textbf{RQ2: How effective are the synthesized profiles at detecting faults in controlled environments?}
  This question tests the accuracy of our method by applying mutation testing \autocite{gomez-abajoMutationTestingTaskOriented2024} to estimate the capacity of the created profiles to detect specific, injected faults.
\item \textbf{RQ3: How effective is the approach at identifying real-world bugs and ensuring task completion in deployed chatbots?}
  This is the practical, real-world applicability of our framework by calculating the Bug Detection Rate (\ac{BDR}) and Task Completion Rate (\ac{TCR}) of the generated profiles against real-world chatbots.
\end{itemize}

\textit{Thesis organisation.}
\autoref{chapter:state_of_the_art} sets up the context and state of the art of chatbot testing.
\autoref{chapter:tracer} lays out the primary methodoly of how \ac{TRACER} extracts models from chatbots.
\autoref{chapter:user-profiles} explains the user profile structure, and the way \ac{TRACER} creates them.
\autoref{chapter:tool_support} illustrates \ac{TRACER} Command Line Interface (CLI) and web application to utilize both Sensei and \ac{TRACER}.
\autoref{chapter:evaluation} provides the comparison of \ac{TRACER} with the research questions.
\autoref{chapter:conclusion} summarizes the thesis and addresses future work.




% Im just keeping this from the template to have it as a reference of how some things work
% \subsection{Subsection}
%
%
% \begin{table}[htpb]
%   \caption[Example table]{An example for a simple table.}\label{tab:sample}
%   \centering
%   \begin{tabular}{l l l l}
%     \toprule
%       A & B & C & D \\
%     \midrule
%       1 & 2 & 1 & 2 \\
%       2 & 3 & 2 & 3 \\
%     \bottomrule
%   \end{tabular}
% \end{table}
%
% \begin{figure}[htpb]
%   \centering
%   % This should probably go into a file in figures/
%   \begin{tikzpicture}[node distance=3cm]
%     \node (R0) {$R_1$};
%     \node (R1) [right of=R0] {$R_2$};
%     \node (R2) [below of=R1] {$R_4$};
%     \node (R3) [below of=R0] {$R_3$};
%     \node (R4) [right of=R1] {$R_5$};
%
%     \path[every node]
%       (R0) edge (R1)
%       (R0) edge (R3)
%       (R3) edge (R2)
%       (R2) edge (R1)
%       (R1) edge (R4);
%   \end{tikzpicture}
%   \caption[Example drawing]{An example for a simple drawing.}\label{fig:sample-drawing}
% \end{figure}
%
% \begin{figure}[htpb]
%   \centering
%
%   \pgfplotstableset{col sep=&, row sep=\\}
%   % This should probably go into a file in data/
%   \pgfplotstableread{
%     a & b    \\
%     1 & 1000 \\
%     2 & 1500 \\
%     3 & 1600 \\
%   }\exampleA
%   \pgfplotstableread{
%     a & b    \\
%     1 & 1200 \\
%     2 & 800 \\
%     3 & 1400 \\
%   }\exampleB
%   % This should probably go into a file in figures/
%   \begin{tikzpicture}
%     \begin{axis}[
%         ymin=0,
%         legend style={legend pos=south east},
%         grid,
%         thick,
%         ylabel=Y,
%         xlabel=X
%       ]
%       \addplot table[x=a, y=b]{\exampleA};
%       \addlegendentry{Example A}
%       \addplot table[x=a, y=b]{\exampleB};
%       \addlegendentry{Example B}
%     \end{axis}
%   \end{tikzpicture}
%   \caption[Example plot]{An example for a simple plot.}\label{fig:sample-plot}
% \end{figure}
%
% \begin{figure}[htpb]
%   \centering
%   \begin{tabular}{c}
%   \begin{lstlisting}[language=SQL]
%     SELECT * FROM tbl WHERE tbl.str = "str"
%   \end{lstlisting}
%   \end{tabular}
%   \caption[Example listing]{An example for a source code listing.}\label{fig:sample-listing}
% \end{figure}
