% !TeX root = ../main.tex
% Add the above to each chapter to make compiling the PDF easier in some editors.

\chapter{Introduction}\label{chapter:introduction}

\section{Section}

The proliferation of conversational agents, commonly refered to as chatbots, has fundamentally transformed the landscape of human-computer interaction accross diverse domains. From general-purpose assistants such as OpenAI's ChatGPT \autocite{ChatGPT} to specialized systems like Google's Gemini \autocite{GoogleGemini}, these agents have become integral to our daily lives, providing users with instant access to information and services through natural language interfaces. The rise of these agents is driven by the increasing demand for efficient and user-friendly interactions, as well as the advancements in artificial intelligence that enable more sophisticated and context-aware responses.
These intelligent systems allow for natural language interaction with services ranging from customer support and e-commerce platforms to municipal services and educational resources.
The proliferation of these agents has been further accelerated by recent advances in generative \ac{AI}, particurly the advance in \acp{LLM}, which have significantly enhanced the capabilities of these chatbots, allowing them to both create and understand natural language without explicitly programmed rules.

The presence of these conversational agents in so many applications has elevated the concerns regarding their reliability, correctness, and overall quality assurance. As these systems appear in domains, such as healthcare or finances, which require high levels of trust, the need for rigorous testing and validation becomes paramount. However, traditional testing is inadequate for these systems, as they often rely on rules that cannot be applied to the complexity of natural language.

\subsection{Subsection}

See~\autoref{tab:sample}, \autoref{fig:sample-drawing}, \autoref{fig:sample-plot}, \autoref{fig:sample-listing}.

\begin{table}[htpb]
  \caption[Example table]{An example for a simple table.}\label{tab:sample}
  \centering
  \begin{tabular}{l l l l}
    \toprule
      A & B & C & D \\
    \midrule
      1 & 2 & 1 & 2 \\
      2 & 3 & 2 & 3 \\
    \bottomrule
  \end{tabular}
\end{table}

\begin{figure}[htpb]
  \centering
  % This should probably go into a file in figures/
  \begin{tikzpicture}[node distance=3cm]
    \node (R0) {$R_1$};
    \node (R1) [right of=R0] {$R_2$};
    \node (R2) [below of=R1] {$R_4$};
    \node (R3) [below of=R0] {$R_3$};
    \node (R4) [right of=R1] {$R_5$};

    \path[every node]
      (R0) edge (R1)
      (R0) edge (R3)
      (R3) edge (R2)
      (R2) edge (R1)
      (R1) edge (R4);
  \end{tikzpicture}
  \caption[Example drawing]{An example for a simple drawing.}\label{fig:sample-drawing}
\end{figure}

\begin{figure}[htpb]
  \centering

  \pgfplotstableset{col sep=&, row sep=\\}
  % This should probably go into a file in data/
  \pgfplotstableread{
    a & b    \\
    1 & 1000 \\
    2 & 1500 \\
    3 & 1600 \\
  }\exampleA
  \pgfplotstableread{
    a & b    \\
    1 & 1200 \\
    2 & 800 \\
    3 & 1400 \\
  }\exampleB
  % This should probably go into a file in figures/
  \begin{tikzpicture}
    \begin{axis}[
        ymin=0,
        legend style={legend pos=south east},
        grid,
        thick,
        ylabel=Y,
        xlabel=X
      ]
      \addplot table[x=a, y=b]{\exampleA};
      \addlegendentry{Example A}
      \addplot table[x=a, y=b]{\exampleB};
      \addlegendentry{Example B}
    \end{axis}
  \end{tikzpicture}
  \caption[Example plot]{An example for a simple plot.}\label{fig:sample-plot}
\end{figure}

\begin{figure}[htpb]
  \centering
  \begin{tabular}{c}
  \begin{lstlisting}[language=SQL]
    SELECT * FROM tbl WHERE tbl.str = "str"
  \end{lstlisting}
  \end{tabular}
  \caption[Example listing]{An example for a source code listing.}\label{fig:sample-listing}
\end{figure}
