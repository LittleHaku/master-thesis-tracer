% !TeX root = ../main.tex
% Add the above to each chapter to make compiling the PDF easier in some editors.

\chapter{Evaluation}\label{chapter:evaluation}

In this section we aim to
evaluate how \ac{TRACER} perfoms at
modelling and synthesising user profiles
from a chatbot under testing.
To do so we will answer the following research questions:

\begin{itemize}
\item \textbf{RQ1: How effective is TRACER in modeling chatbot functionality?}
  This question evaluates the capability of our model discovery method to attain high functional coverage in a controlled environment where the ground truth is available.
\item \textbf{RQ2: How effective are the synthesised profiles at detecting faults in controlled environments?}
  This question tests the accuracy of our method by applying mutation testing \autocite{gomez-abajoMutationTestingTaskOriented2024} to estimate the capacity of the created profiles to detect specific, injected faults.
\item \textbf{RQ3: How accurately does TRACER model the functionalities of real-world, deployed chatbots?}
  This question addresses the practical, real-world applicability of \ac{TRACER}.
  To answer it, we will compare the inferred model against a handmade one
  of deployed chatbots
  and measure the precission, recall and F1-score.
\end{itemize}

RQ1 evaluates the coverage of our approach
in terms of activating chatbot functionalities.
To measure the coverage, \ac{TRACER} will be executed
against the chatbot under testing
and the generated synthesised profiles
will be executed with SENSEI against the same chatbot.
We will perform this with chatbots created with the Taskyto framework
because this framework offers the possibility
of reporting the activated modules.
With these reports we will be able to measure
the coverage obtained during the \ac{TRACER} and SENSEI executions independently.

We anticipate higher coverage during SENSEI profile execution,
as profiles are designed to systematically test all discovered parameters
and their valid combinations.
For instance, \ac{TRACER} may find that a pizzeria chatbot
offers six pizza types, but does not try them all during exploration;
instead, the generated profiles are designed to test all options.

Evaluating RQ1 is crucial for understanding
how thoroughly \ac{TRACER} exercises the chatbot’s functionality.
High coverage does not guarantee high quality,
but it implies that a significant part of a system has been tested,
increasing the confidence in its reliability
\autocite{ammannIntroductionSoftwareTesting2017}.


RQ2 assesses the practical effectiveness
of our approach in detecting chatbot errors.
To this aim, we employ mutation testing
\autocite{demilloHintsTestData1978}
by introducing defects into correct chatbots to create mutants,
and testing whether the generated profiles detect these defects.
A profile successfully detects a mutation (i.e., a fault)
when the simulated user fails to complete a goal
(e.g., ordering a pizza type that was removed in the mutant)
or when an expected output is missing or incorrect
This way, assessing RQ2 is important
to provide evidence that our approach produces
profiles that are sufficiently comprehensive
to reveal real-world chatbot errors.

RQ3 assesses \ac{TRACER}'s ability at modeling and synthesising
real-world deployed chatbots.
Contrary to RQ1 and RQ2 that work with chatbots developed with \ac{TRACER} framework
specifically for this evaluation,
RQ3 evaluates chatbots deployed for real-world applications
such as, universities, townhalls or transportation companies.
To evaluate this, we will craft a hand-made model
by interacting with the chatbot
and then compare it with the one inferred by \ac{TRACER}.
