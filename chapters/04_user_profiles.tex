% !TeX root = ../main.tex
% Add the above to each chapter to make compiling the PDF easier in some editors.

\chapter{User Profile Structure and Generation}\label{chapter:user_profiles}

Once the model has been finished,
we use it to automatically generate user profiles for SENSEI.
These serve as test cases designed to verify
the discovered functionalities of the chatbot,
its handling of different inputs,
to check if the outputs match the expected value,
and to find other errors such as timeouts.

First, \autoref{sec:profile-structure}
will cover the structure of these profiles and how they work,
then \autoref{sec:profile-generation}
will cover how these profiles are generated from the infered model.


\section{User Profiles Structure}\label{sec:profile-structure}

A user profile contains all the information
that characterises the user,
the conversation goals,
interaction style,
and other information such as the \ac{LLM} that will be used,
or the number of conversations and turn per conversations.
These profiles are structured in a YAML file,

\begin{figure}[htpb]
  \centering
  \lstinputlisting[language=yaml]{code/pizza_order_placement.yaml}
  \caption[Example of a conversation profile for a pizzeria chatbot.]{Example of a conversation profile for a pizzeria chatbot.}\label{lst:user_profile}
\end{figure}

\section{User Profiles Generation}\label{sec:profile-generation}
