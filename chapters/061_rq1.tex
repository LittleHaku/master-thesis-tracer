\section{RQ1: Coverage of Chatbot Functionality}

The primary goal of this first experiment
is to quantitatively measure the coverage
of \ac{TRACER} during its execution and
the subsequent execution of the generated user profiles with SENSEI.
To achieve this white-box evaluation
we ran the experiment in a controlled environment
using a chatbot whose source code is known,
which allowed us to measure the number of modules activated
during each conversation.

\subsection{Experiment Setup}

To assess the effectiveness of the proposed approach in discovering chatbot functionality,
we applied \ac{TRACER} to four deployed Taskyto chatbots
\autocite{sanchezcuadradoAutomatingDevelopmentTaskoriented2024}.
We chose chatbots built this technology because their code is open-source
\autocite{SatorichatbotsTaskyto2025},
which allowed us to instrument them
to trace the modules activated during conversations,
as required to assess RQ1.
Moreover, the declarative structure of the chatbots
facilitated injecting faults into them, as required to assess RQ2.

Taskyto is a declarative framework
to develop LLM-based task-oriented chatbots using a YAML-like DSL.
Chatbot definitions consist of any number of modules of five possible types:

\begin{itemize}
  \item 
    \textbf{Menu:} to define conversation alternatives.
\item
  \textbf{Sequence:} to define a sequence of conversation steps.
\item
  \textbf{Action:} to execute Python code and produce a
verbatim or rephrased response upon receiving some input data.
\item
  \textbf{Data gathering:} to request user input data.
\item
  \textbf{Question answering:} to declare a FAQ.
\end{itemize}


\autoref{tab:rq1_chatbots} displays size metrics
of the chatbots used for this experiment.
We show the number of modules, lines of code (YAML lines),
input fields or parameters,
the number of possible values for these inputs in case that they are an enumeration,
and the number of question-answer pairs for the FAQ.

\textit{Bike-shop} schedules bike repair appointments
and answers bike maintenance questions.
\textit{Photography} is a chatbot for a photography shop,
which answers questions about the shop,
gathers contact details of clients, and gives price estimates.
\textit{Pizza-order} handles pizza orders,
including their size, toppings and drinks.
\textit{Veterinary} sets appointments and answers questions about a veterinary clinic.

\begin{table}[htpb]
\centering
\caption{Size metrics of the four Taskyto chatbots used in the evaluation.}
\label{tab:rq1_chatbots}
\begin{tabular}{@{}lccccc@{}}
\toprule
\textbf{Chatbot} & \textbf{Modules} & \textbf{YAML lines} & \textbf{Inputs} & \textbf{Values} & \textbf{Q\&A Pairs} \\ \midrule
Bike-shop & 3 & 65 & 3 & 2 & 4 \\
Photography & 5 & 140 & 9 & 12 & 5 \\
Pizza-order & 10 & 282 & 22 & 78 & 6 \\
Veterinary & 3 & 71 & 3 & 4 & 5 \\ \bottomrule
\end{tabular}
\end{table}

We modified Taskyto to generate logs
for each conversation containing
each activated module,
expected input data from the user (e.g., pizza type),
values provided for the input data with a limited set of options (e.g., Carbonara),
and performed questions from the chatbot FAQ.
Subsequently, a Python script was used to unify these logs,
first by merging them into a unified log,
and then by generating a Markdown report
with the coverage percentage obtained for each chatbot.

To account for the non-deterministic nature of \acp{LLM},
we repeated the experiment three times.
To ensure that \ac{TRACER} finds as many functionalities as possible
we ran TRACER for 20 conversations with 12 turns each
to allow sufficient time for functionalities to branch out.
These parameters provide a balance between discovery and time.
The \ac{LLM} used by \ac{TRACER} was Google's Gemini 2.0 Flash
with the default model's parameters (1.0 temperature),
which we did not modify since it allowed creativity in the conversations
while still following \ac{TRACER}'s instructions.

SENSEI and Taskyto used OpenAI's GPT-4o-mini,
we used an OpenAI model
since these programs did not yet support a different provider
for Taskyto, the temperature was set to 0
to ensure the chatbot's behaviour was deterministic.
SENSEI used the temperatures defined in the generated profiles,
which as explained during the profile generation in \autoref{sec:profile-generation}
is set randomly.

The overall procedure then, is as follows.
First, we executed TRACER three times with the aforementioned settings
measuring the Taskyto coverage during each execution.
Using TRACER's generated profiles, we then ran SENSEI.
Since TRACER was run three times,
this resulted in three different sets of profiles.
Consequently, we measured the coverage for each of these three executions
and also reported an aggregated coverage.



\subsection{Results and Discussion}

\autoref{tab:rq1_coverage_results}
summarizes the results obtained throughout this experiment.
It contains the median and aggregated coverage obtained
during TRACER's exploration alone
and during the execution of the generated profiles within SENSEI.
The coverages are split into
modules, input parameters, values of these inputs and FAQ questions.
The table highlights in green the highest median and aggregate coverage
obtained per chatbot and metric
(i.e., module, input, value or question).

\begin{table}[htpb]
\centering
\caption{Coverage of TRACER (chatbot exploration) and SENSEI (profile execution).
In green there is the greatest coverage obtained
median and aggregate coverage per chatbot and metric
(i.e., module, input, value or question).
}
\label{tab:rq1_coverage_results}

% Define a custom color for the green highlight
\definecolor{lightgreen}{HTML}{C9FBC9}
\definecolor{stronggreen}{HTML}{53E753}

\begin{tabular}{lccccc}
\toprule
\textbf{Stat.} & \textbf{Tool} & \textbf{Module (\%)} & \textbf{Input (\%)} & \textbf{Value (\%)} & \textbf{Question (\%)} \\ \midrule

% --- Bike-shop Data ---
\rowcolor{gray!10} \multicolumn{6}{c}{\textbf{Bike-shop}} \\
\raisebox{-0.5\normalbaselineskip}[0pt][0pt]{Median} & TRACER & \cellcolor{lightgreen}100 & \cellcolor{stronggreen}85.71 & \cellcolor{stronggreen}83.33 & \cellcolor{stronggreen}75.00 \\
& SENSEI & \cellcolor{lightgreen}100 & 71.43 & 50.00 & 50.00 \\ \addlinespace
\raisebox{-0.5\normalbaselineskip}[0pt][0pt]{Aggregate} & TRACER & \cellcolor{lightgreen}100 & \cellcolor{lightgreen}85.71 & \cellcolor{lightgreen}83.33 & \cellcolor{stronggreen}75.00 \\
& SENSEI & \cellcolor{lightgreen}100 & \cellcolor{lightgreen}85.71 & \cellcolor{lightgreen}83.33 & 50.00 \\ \midrule

% --- Photography Data ---
\rowcolor{gray!10} \multicolumn{6}{c}{\textbf{Photography}} \\
\raisebox{-0.5\normalbaselineskip}[0pt][0pt]{Median} & TRACER & \cellcolor{lightgreen}100 & 73.33 & \cellcolor{stronggreen}64.71 & 20.00 \\
& SENSEI & \cellcolor{lightgreen}100 & \cellcolor{stronggreen}80.00 & 58.82 & \cellcolor{stronggreen}40.00 \\ \addlinespace
\raisebox{-0.5\normalbaselineskip}[0pt][0pt]{Aggregate} & TRACER & \cellcolor{lightgreen}100 & 73.33 & 76.47 & 20.00 \\
& SENSEI & \cellcolor{lightgreen}100 & \cellcolor{stronggreen}93.33 & \cellcolor{stronggreen}94.12 & \cellcolor{stronggreen}80.00 \\ \midrule

% --- Pizza-order Data ---
\rowcolor{gray!10} \multicolumn{6}{c}{\textbf{Pizza-order}} \\
\raisebox{-0.5\normalbaselineskip}[0pt][0pt]{Median} & TRACER & 83.33 & 67.86 & 27.38 & \cellcolor{lightgreen}100 \\
& SENSEI & \cellcolor{lightgreen}100 & \cellcolor{stronggreen}96.43 & \cellcolor{stronggreen}69.05 & \cellcolor{lightgreen}100 \\ \addlinespace
\raisebox{-0.5\normalbaselineskip}[0pt][0pt]{Aggregate} & TRACER & \cellcolor{lightgreen}100 & \cellcolor{lightgreen}100 & 48.81 & \cellcolor{lightgreen}100 \\
& SENSEI & \cellcolor{lightgreen}100 & \cellcolor{lightgreen}100 & \cellcolor{stronggreen}90.48 & \cellcolor{lightgreen}100 \\ \midrule

% --- Veterinary Data ---
\rowcolor{gray!10} \multicolumn{6}{c}{\textbf{Veterinary}} \\
\raisebox{-0.5\normalbaselineskip}[0pt][0pt]{Median} & TRACER & \cellcolor{lightgreen}100 & 50.00 & \cellcolor{lightgreen}44.44 & 20.00 \\
& SENSEI & \cellcolor{lightgreen}100 & \cellcolor{stronggreen}62.50 & \cellcolor{lightgreen}44.44 & \cellcolor{stronggreen}40.00 \\ \addlinespace
\raisebox{-0.5\normalbaselineskip}[0pt][0pt]{Aggregate} & TRACER & \cellcolor{lightgreen}100 & 62.50 & 55.56 & 40.00 \\
& SENSEI & \cellcolor{lightgreen}100 & \cellcolor{stronggreen}87.50 & \cellcolor{stronggreen}77.78 & \cellcolor{stronggreen}80.00 \\ \bottomrule
\end{tabular}
\end{table}

Regarding activated modules
both \ac{TRACER} and SENSEI achieved a 100\% aggregate coverage
across all chatbots,
and similarly for the median coverage, except for one case where
\ac{TRACER} achieved an 83.33\%.
This shows that the core functionalities of all chatbots
were successfully discovered during the exploration
and exercised during the profile execution.

In terms of the input parameters and
the values that these input fields allow,
we also obtained high values,
with a minimum value of 62.50\% aggregated value for TRACER
and a 85.71\% for SENSEI,
with both reaching a 100\% for the \textit{Pizza-order} chatbot.
For the values that these fields can take, we obtained
coverages as low as 48.81\% or 55.56\% in the aggregate \ac{TRACER} coverage,
but then for SENSEI we obtained higher values between 77.78\% and 94.12\%.
This was expected since during the exploration
TRACER might find values or options and not explore them further
but they are subsequently used in the profiles,
which means that SENSEI will use these values.
A clear example of this is the aggregate coverage in in the \textit{pizza-order} values,
with only a 48.81\% during exploration but a 90.48\% for the profiles,
This example shows a greater jump since as shown in \autoref{tab:rq1_chatbots}
this chatbot has the highest number of values
(78 values, while the other chatbots have 2, 12 and 4).
This shows that the profile creation is comprehensive.

Lastly, aggregate question coverage was somewhat lower,
with SENSEI oscillating between a 50 and a 100\%.
This is the only metric where the aggregate coverage of TRACER
was higher than SENSEI's, which is contrary to the expected outcome.
This occurred because Taskyto sometimes struggles with FAQs;
if the question is not asked exactly as it is written,
Taskyto often fails to identify it.
\ac{TRACER} asked it as it was written in the FAQ
"Which is the price of a new tire?"
while SENSEI used "What is the price of a new tire?"
and Taskyto did not proceed with the answer, thus,
not adding it to the coverage logs.

\subsection{Answer to RQ1}

We conclude that \ac{TRACER} is able to effectively identify
most of a chatbot's functionalities, input parameters and values for these inputs.
Furthermore, we can conclude that TRACER effectively models the chatbot
since the inferred model used to generate the user profiles for SENSEI
managed to achieve aggregated coverages in the 77.78\% to the 100\% range
except for the outlier in the \textit{Bike-shop} FAQ.
