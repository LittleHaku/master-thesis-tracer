\section{RQ2: Effectiveness in Detecting Faults}

After establishing \ac{TRACER}'s effectiveness at
discovering chatbot functionalities (RQ1),
this second experiment aims to assess
the practical fault detection capability
of the synthesised user profiles.
To this end, we will use mutation testing
\autocite{demilloHintsTestData1978}
to create faulty versions of the chatbots (mutants)
and measure the percentage of these faults
that are detected when executing the generated user profiles with SENSEI.

\subsection{Experiment Setup}

We applied mutations (i.e., injected errors)
to the four chatbots used in the previous RQ.
To choose the user profiles,
we selected the set that achieved the highest coverage
out of the three sets generated during RQ1.

To create the mutants we used a Python script
that generates as many mutant chatbots as possible from a given original,
introducing a single error in each mutant.
It achieves this by systematically injecting
single faults into the chatbot's YAML definition.
The following mutation operators are based on previous work
on mutation testing for task-oriented chatbots
\autocite{gomez-abajoMutationTestingTaskOriented2024, urricoMutaBotMutationTesting2024}:

\begin{itemize}
  \item \textbf{Delete Enum Value:}
    deleting a possible value from an input parameter
    (e.g., removing \texttt{small} from \texttt{pizza\_size}).
  \item \textbf{Change Optionality:}
    making an optional parameter required, or vice-versa.
  \item \textbf{Delete QA Pair:}
    removing a question from a FAQ module.
  \item \textbf{Swap QA Answer:}
    swapping the answers of two questions.
  \item \textbf{Delete Menu Alternative:}
    removing a conversational choice from a menu module.
  \item \textbf{Delete Fallback:}
    removing the fallback response.
  \item \textbf{Delete Sequence Step:}
    removing a necessary step from a workflow.
  \item \textbf{Swap Sequence Steps:}
    swapping the order of two steps in a sequence.
  \item \textbf{Delete Output Data:}
    omitting a piece of data from a chatbot's response (e.g., the price).
  \item \textbf{Change Rephrase Behavior:}
    altering the LLM's rephrasing mode.
  \item \textbf{Change Memory Scope:}
    changing the chatbot's ability to access data from previous turns.
\end{itemize}

The script systematically injects faults in every possible location of the chatbots.
However, due to the time and cost of running SENSEI,
we randomly selected two mutants of each type for each chatbot.
If only one or zero mutants of a type existed, all were selected.

A mutant is considered "killed"
if, during the execution of the SENSEI profiles,
at least one of the following conditions is met:
\begin{itemize}
  \item The simulation results in a crash, timeout, or conversation loop.
  \item A specified conversation \texttt{goal} is not achieved
    (e.g., the user fails to order the pizza).
  \item An expected \texttt{output} is missing or incorrect.
\end{itemize}


\subsection{Results and Discussion}

\begin{table}[htpb]
\centering
\caption{Summary of the mutation analysis results for each chatbot.}
\label{tab:rq2_mutation_results}
\begin{tabular}{@{}lcccccc@{}}
\toprule
\textbf{Chatbot} & \makecell[c]{\textbf{\# Mutants}} & \makecell[c]{\textbf{\# Profiles}} & \makecell[c]{\textbf{\# Convs.}\\\textbf{Per Chatbot}} & \makecell[c]{\textbf{Total}\\\textbf{Convs.}} & \makecell[c]{\textbf{Mutation}\\\textbf{Score (\%)}} & \makecell[c]{\textbf{\# Live}\\\textbf{Mutants}} \\ \midrule
Bike-shop & 13 & 3 & 9 & 117 & 91.0 & 1 \\
Photography & 18 & 3 & 12 & 216 & 76.9 & 3 \\
Pizza-order & 20 & 4 & 22 & 440 & 75.0 & 3 \\
Veterinary & 13 & 5 & 21 & 273 & 91.7 & 1 \\ \midrule
\textbf{TOTAL} & \textbf{64} & \textbf{15} & \textbf{64} & \textbf{1046} & \textbf{84.6} & \textbf{8} \\ \bottomrule
\end{tabular}
\end{table}

\autoref{tab:rq2_mutation_results}
shows the results for each chatbot.
The column \texttt{\# Mutants} represents
the total number of mutants generated
following the rule of selecting a maximum of 2 per mutation type.
\texttt{\# Profiles} shows the amount of profiles used
followed by \texttt{\# Convs. Per Chatbot},
which represents the total number of conversations
that occur by using the set of profiles from the previous column.
\texttt{Total Convs.} is the product of
\texttt{\# Mutants} and \texttt{\# Convs. Per Chatbot}
since each mutant is a new chatbot.
Finally, \texttt{Mutation Score (\%)} and \texttt{\# Live Mutants}
show the results from the executions.
\ac{MS} is the percentage of the mutants that were killed,
while the number of live mutants
represents the number of errors that were not found.

We observe a high overall \acl{MS}
with an average \ac{MS} of 84.6\%.
All chatbots achieved at least 75.0\%,
and peaking at 91.7\%.
The cases with the lowest \ac{MS}
(75.0\% and 76.9\%)
correspond to the most complex chatbots
(see \autoref{tab:rq1_chatbots}).

Although most of the errors were detected automatically (58\%),
some cases required manually inspecting the conversation,
e.g., when the chatbot omitted a specific piece of data
(such as the location of the photography shop)
but included other requested information.
For instance, the Photography chatbot outputs a summary of the shop information,
and some but not all information is set in this output,
therefore, there is no error of an unachieved goal,
even if the response is incomplete.
Detecting such errors automatically
would require testing rules that set oracles
and analyse the expected chatbot output values,
as explained in \autocite{delaraSensei}.

\subsection{Threats to Validity}
A threat to construct validity
concerns the set of mutation operators used.
While based on previous work
and designed to cover Taskyto's declarative elements,
they may not represent
all possible types of real-world faults.
Consequently, the mutation score
is a strong indicator
but not an exhaustive measure
of the profiles' fault-detection capabilities on other platforms.

Furthermore,
a potential threat to internal validity
lies in the manual inspection process required
to identify false negatives and equivalent mutants.
Although necessary for accuracy,
this step introduces the possibility of human error.

\subsection{Answer to RQ2}

The high mutation scores achieved (84.6\% on average)
provide strong evidence
that the profiles synthesised by TRACER are effective
at detecting faults in controlled environments.
However, full automation would require
complementing the profiles with (manually created) testing rules
that search for specific data in the profile outputs.

