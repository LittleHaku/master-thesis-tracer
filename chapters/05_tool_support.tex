% !TeX root = ../main.tex
% Add the above to each chapter to make compiling the PDF easier in some editors.

\chapter{Tool Support}\label{chapter:tool_support}

To ensure that the previous methodology can be reproduced
we have implemented \acf{TRACER} in an open-source Python package
to allow users to execute it from a \ac{CLI}.
On top of that, we have developed a web application
that allows users to execute both \ac{TRACER} and Sensei
without the need of knowing how to operate the command line.


\section{Implementation and Architecture}

\subsection{Core Framework: LangGraph}

As explained during the \autoref{sec:profile-generation} (\nameref{sec:profile-generation})
\ac{TRACER} relies on \acp{LLM},
this is why used LangGraph \autocite{LangGraph} as our framework for development.
The reason why we chose LangGraph
was because it allows to manage and orchestrate
complex agentic workflows with states,
it is also an industry standard and
it has an extensive documentation.
LangGraph allows us to orchestrate the different stages
and to keep complex states where we store
the inferred model and the fields generated for the profiles.
Right now \ac{TRACER} allows OpenAI and Gemini models,
but thanks to LangGraph it would be easy to add other \ac{LLM} providers.

\subsection{Modular Architecture}

\ac{TRACER} can infer a chatbot model as long as
the chabots is accessible through an interface, typically a \acs{REST} \acs{API}.
Right now it provides access to communicate with chatbots made with different technologies,
such as Taskyto \autocite{sanchezcuadradoAutomatingDevelopmentTaskoriented2024}, Rasa \autocite{Rasa2020} or 1MillionBot \autocite{1MillionBot}.
In addition, new connectors could be added by extending the current implementation.

Apart from these connectors,
\ac{TRACER} is divided into three modules,
each corresponding to a phase of the methodology:

\begin{itemize}
  \item \textbf{Explorer Module:}
    Contains the Explorer Agent
    and implements the logic for the Exploration Phase (see \autoref{sec:exploration}),
    managing the conversational sessions and the initial extraction of Functionality Nodes.

  \item \textbf{Refinement Module:}
    Implements the logic for the Refinement Phase (see \autoref{sec:refinement}),
    responsible for consolidating functionalities,
    classifying the chatbot, and inferring the final workflow structure.

  \item \textbf{Profile Generation Module:}
    Implements the seven-step synthesis process (see \autoref{sec:profile-generation}),
    taking the final chatbot model and generating the YAML user profiles.
\end{itemize}

\subsection{Generated Artifacts}

Upon completion, \ac{TRACER} generates the following artifacts
containing the results from the full analysis performed on the target chatbot:

\begin{itemize}
  \item A set of user profiles
    representing realistic users that would use the application
    and that will act as test cases
    (see \autoref{code:yaml-profile-pizza} and \autoref{code:yaml-profile-drinks})
  \item A markdown report containing the inferred model information like
    the discovered functionalities, fallback message, language
    and also other information like token usage, number of \ac{LLM} calls or estimated cost.
  \item A graph representing the inferred model's workflow (see \autoref{fig:pizzeria-workflow})
  \item A JSON file containing the same workflow but in text.
\end{itemize}



\section{Distribution and Development Workflow}

Before detailing the \ac{CLI}'s functioning,
this section will describe \ac{TRACER}'s packaging, distribution,
and the software engineering practices used to maintain its quality.
\ac{TRACER} is packaged and distributed as a package on the \acf{PyPI} repository
(\url{https://pypi.org/project/chatbot-tracer/}),
making it easy to install by running
\texttt{pip install chatbot-tracer}.
This not only makes it easy to use,
but also makes it easy to implement into other projects
such as the web application done,
or other projects that could use \ac{TRACER}
since it can just be added as another package requirement.

To ensure code quality and automate the release process
\ac{TRACER} makes use of GitHub Actions for the \ac{CI/CD} pipeline.
For the \ac{CI} we made use of Ruff \autocite{Ruff}.
Ruff is Python linter and formatter written in Rust
that combines tools like Flake8 or Black into a single and faster tool.
We made use of Ruff not only to enforce a consistent code style,
but also to enforce code quality standards,
such as ensuring proper documentation
and managing code complexity by setting thresholds for metrics like McCabe's cyclomatic complexity.
For the \ac{CD} side,
we implemented a pipeline that whenever a tag with the format \texttt{v*.*.*} is published
automatically builds the package,
publishes it to \ac{PyPI},
and creates the corresponding GitHub release.
All the \ac{TRACER} source code can be accessed in \url{https://github.com/Chatbot-TRACER/TRACER}.
