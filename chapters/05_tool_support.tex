% !TeX root = ../main.tex
% Add the above to each chapter to make compiling the PDF easier in some editors.

\chapter{Tool Support}\label{chapter:tool_support}

To ensure that the previous methodology can be reproduced
we have implemented \acf{TRACER} in an open-source Python package
to allow users to execute it from a \ac{CLI}.
On top of that, we have developed a web application
that allows users to execute both \ac{TRACER} and Sensei
without the need of knowing how to operate the command line.

\section{Distribution and Development Workflow}

Before detailing the \ac{CLI}'s functioning,
this section will describe \ac{TRACER}'s packaging, distribution,
and the software engineering practices used to maintain its quality.
\ac{TRACER} is packaged and distributed as a package on the \acf{PyPI} repository
(\url{https://pypi.org/project/chatbot-tracer/}),
making it easy to install by running
\texttt{pip install chatbot-tracer}.
This not only makes it easy to use,
but also makes it easy to implement into other projects
such as the web application done,
or other projects that could use \ac{TRACER}
since it can just be added as another package requirement.

To ensure code quality and automate the release process
\ac{TRACER} makes use of GitHub Actions for the \ac{CI/CD} pipeline.
For the \ac{CI} we made use of Ruff \autocite{Ruff}.
Ruff is Python linter and formatter written in Rust
that combines tools like Flake8 or Black into a single and faster tool.
We made use of Ruff not only to enforce a consistent code style,
but to ensure that it is properly documented,
or that a function is not too complex (by checking the McCabe complexity).
For the \ac{CD} side,
we implemented a pipeline that whenever a tag with the format \texttt{v*.*.*} is published
automatically builds the package,
publishes it to \ac{PyPI},
and creates the corresponding GitHub release.
All the \ac{TRACER} source code can be accessed in \url{https://github.com/Chatbot-TRACER/TRACER}.
